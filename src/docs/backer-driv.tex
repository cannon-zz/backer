\documentclass{article}
\usepackage{fullpage}
\usepackage{graphics}

\title{Additional Info for the Backer 16/32 Device Driver}
\author{Kipp Cannon\footnote{\texttt{kcannon@users.sourceforge.net}}}
\date{\today}

\begin{document}

\maketitle

\section{Programming}

\subsection{Tape format}
\label{prog:format}

If you want to play around with the tape format parameters, here's an
explanation of their function
\begin{itemize}
\item
In \texttt{backer\_fmt.h}
\begin{description}
\item[\texttt{BKR\_LEADER}]
The bit pattern to fill the leader of the video field with.  Do not use
\texttt{0x00}.

\item[\texttt{BKR\_TRAILER}]
The bit pattern to fill the trailer of the video field with.  Do not use
\texttt{0x00}.

\item[\texttt{BKR\_FILLER}]
The bit pattern to fill unused portions of the video field with.  Do not
use \texttt{0x00}.

\item[\texttt{KEY\_LENGTH}]
The number of bytes in the sector key.

\item[\texttt{BOR\_LENGTH}]
The number of seconds for which the beginning of record mark should last.
This must be long enough for a VCR to synchronise itself to the video
signal.  I have found that 4 seconds provides a good safety margin for my
own equipment but perhaps, for a more ``generic'' value, it might be better
to set this to a longer time.

\item[\texttt{EOR\_LENGTH}]
The number of seconds for which the end of record mark should last.  The
only real restriction here is that this must not be 0.  A minimum of two
EOR sectors are required in order to guarantee that the driver reports EOF
to the operating system (an EOR sector causes the internal buffer to be
flushed unless it's already empty in which case EOF is returned) but more
can be generated to provide a better visual display for a human operator.
My current preference is for 1 second's worth.
\end{description}

\item
In \texttt{backer\_fmt.c}
\begin{description}
\item[\texttt{CORRELATION\_THRESHOLD}]
The threshold that must be reached in order for the sector locator
algorithm to decide that it has found the sector key.  If you need to
adjust this, run \texttt{bkrmonitor} on a full tape of data and set the
threshold between the ``Worst match'' and ``Closest non-match'' values.

\item[\texttt{format[]}]
An array of parameters which describe the data format for each operating
mode.  The leader size specifies the number of bytes to set aside for the
leader at the start of each sector;  the trailer, the number of bytes for
the trailer at the end of each sector.  The interleave specifies the number
of blocks in each sector (and, thus, the number of bytes separating
sequential bytes of each block).  The parity size gives the number of
parity bytes in each block of data.  The leader and trailer lengths should
both be multiples of the number of bytes in each line of video in the
selected mode.  The parity length should be an even number.  After
subtracting the leader and trailer from the sector, the size of each block
is found by dividing the number of bytes remaining by the interleave.  This
must be an integer.  The block size must be large enough to hold its parity
bytes and the sector key (not the whole sector header).  Subject to these
restrictions, you are free to adjust these numbers to anything you desire.
\end{description}

\end{itemize}
\textbf{NOTE:}  when playing around with these numbers, do not rebuild the
driver itself unless you really know what you're doing because you can
damage your system if you put the wrong numbers in.  If you want to play
around with new tape formats, then here's how to do it safely.  First, go
ahead and edit the driver formating code all you want.  Then, rather than
rebuilding the driver, build \texttt{bkrencode} (see the utilities package)
using the modified driver sources.  Test out your changes by doing all your
I/O via \texttt{bkrencode} and the appropriate raw device.  This way your
data processing is done in user space so if you make a mistake and get a
segmentation fault you won't trash your system.  Once you've convinced
yourself that your changes work then you can go ahead and rebuild the
driver to incorporate them.


\section{Tape Format}

\subsection{My Format}

The details of the data format are intimately tied to the practical reality
of the tapes being \emph{video} tapes.  The data stream is naturally
divided into sectors with each sector corresponding to the contents of one
field of video (\(\frac{1}{2}\) a frame).  Each sector has three major
components: a leader, a trailer, and an interleaved data region which
caries the sector's information content.  A diagram of the sector layout is
shown in Figure \ref{fig1}.
\begin{figure}
\begin{center}
\begin{picture}(250,190)
\put(0,0){\framebox(250,25)[c]{trailer}}
\multiput(25,30)(25,0){9}{\line(0,1){130}}
\multiput(0,32)(0,2){64}{\line(1,0){250}}
\put(0,30){\framebox(250,130)[c]{useable}}
\put(0,165){\framebox(250,25)[c]{leader}}
\end{picture}
\end{center}
\caption{The sector format for high data rate shown (not to scale) as it
appears on a television set.  The order in which bytes are written to tape
is left-to-right top-to-bottom.  The section in the middle is the fraction
of the sector available for data storage although some of it must be used
for parity bytes and a sector header.  In low data rate mode there are 4
bytes per line rather than the 10 shown here.}
\label{fig1}
\end{figure}

The leader functions as a cushion giving the VCR's video head a chance to
stabilize onto the video track during playback.  VCRs with poor tracking
may need larger leaders for this purpose.  The trailer's role is to provide
a cushion at the end of the video field to accommodate the loss of video
lines during playback.  VCRs seem to not generate an exact number of video
lines in each frame and I have found my own equipment to fail to playback
as many as 6 of the last lines of the video field.

The rest of the sector is set aside for useable information.  Since video
tapes are noisy environments, an error control coding scheme must be
implemented.  The particular scheme used here is an interleaved
Reed-Solomon code.  The useable portion of the sector is logically divided
into blocks containing data and parity bytes.  The parity bytes are
computed using a reduced Reed-Solomon code and the parity bytes from all
blocks are grouped together at the top of the sector.

At the bottom end of the sector's data area is the sector header which
contains a 22 bit sequence number, 5 bits identifying the type of data in
the sector, a flag indicating whether or not the sector is truncated, the 4
high bits of the number of bytes stored in the sector, and the sector
``key''.  The sequence number is used to skip over repeated sectors during
playback and identify buffer over-runs.  The 5 bit type can be one of three
values indicating that the sector is a BOR marker, and EOR marker or
contains data (this field only needs 2 bits but it is expanded to occupy
all of the unused bits so that future format extensions will appear as
unrecognized sector types to older versions of the driver).

If the sector does not contain its full complement of data (eg.\ it is the
last sector in a file) then the truncate flag in the header is set and the
low 8 bits of the count of bytes in the sector is stored in the last byte
of the data area while the upper 4 bits are stored in the header.

Because the number of lines played back for each video field is not a
constant, it is necessary to implement a mechanism by which sectors can be
found in the byte stream.  This is done by placing a specific sequence of
bytes, called the sector key, into each sector.  By searching for this byte
sequence during playback one can identify the location of a sector.
Specifically, the key is written into the sector as part of the sector
header.

The Backer hardware is not capable of decoding a video signal that contains
long runs of 0's.  We wish to be able to record arbitrary data on the tape,
so to reduce the risk of runs of 0's being written the data area of each
sector is mixed with pseudo-random numbers generated using the sector
number as a seed.  This is done before the sector header is written into
the sector which leaves it unaffected and thus recoverable without having
to know the sector number.

Finally, since video tape is not a high quality data environment it is
necessary to provide error control coding.  After the sector has been
prepared (all of the blocks have been written, it's been randomized and the
header has been inserted) the sector is fed through a Reed-Solomon encoder
block-by-block to generate the parity symbols.

After being prepared, the useable portion of the sector is interleaved onto
the tape to distribute burst errors equitably among the blocks.


\subsection{Danmere's Format}

I myself am not interested in reverse engineering Danmere's tape format as
I have no need to exchange data with Windows machines.  However, if someone
else is interested in doing the grunt work of deciphering the format then I
would be more than happy to implement a Danmere-compatible back end like
\texttt{bkrencode} if I was given the needed information.  To those who are
interested, I have already taken a little stab at it and have accumulated
some tidbits that might be helpful clues.  To assist you in cracking the
code, I should point out that this driver is perfectly capable of reading
tapes written with Danmere's Windows software:  put the driver into raw
mode, play one of Danmere's tapes into it and dump the data to disk.  Then
use your favourite binary file viewer and go crazy.


\section{Error Control}

Here are some notes on the ECC scheme I've used. There are three types of
errors that can occur in this environment and, in increasing order of
severity, they are (i) signal drop outs (ii) framing errors and (iii)
buffer over-run errors.

Signal drop outs are caused mostly by tape defects and power line noise
(hair dryers being turned on and off, etc.) although they could also, in
principle, be caused by hardware glitches (eg.\ excessive phase drift in
the shift clock).  Because the drop outs tend to result in bursts of lost
data, their effect on the data stream is reduced by interleaving the bytes.
Analysis of the noise I see with my own equipment has shown that it is
generally confined to a single line of video.  In every operating mode, the
interleaving I use spreads bytes out farther than this.  A signal drop out
then almost never appears as more than a single symbol error in each of the
blocks it effects --- something easily handled with an error control code.

The error control code used is a reduced Reed-Solomon code using 8 bit
symbols.  For NTSC video in both high and low densities, 8 parity symbols
are generated for each block of 119 bytes.  That means 4 single byte errors
can be corrected in each block of data.  This appears to be plenty as I
have never seen more than 2 on a decent recording but it is much less than
Danmere's default setting of 6 parity bytes in each block of 26.  Danmere
would appear to feel that high interleave ratios are advantageous.

A framing error occurs when the key sequence of a sector is not recognized
and/or a false one recognized in its place.  When a framing error occurs
there are two possibilities.  The first possibility is that tape noise
corrupts the key sequence to the point that it is not recognized by the
driver.  If that's all that happens, then the sector is simply skipped and
a framing error is recorded.  The second possibility is that some part of
the sector other than the key happens to look sufficiently like the key
that the driver starts a sector read-out at the wrong location (this might
be combined with the first case).  If this occurs, since data is being
extracted out of the wrong part of the data stream, none of the blocks will
decode properly in the Reed-Solomon decoder.  Each sector containing an
uncorrectable block is discarded and an error code is reported to the
system.  In both cases, some quantity of data is lost and the first sector
to decode correctly after the error will appear to be out of sequence
causing the driver to also record a buffer over-run.

The third type of error is the buffer over-run.  This occurs when the
computer is not able to process the data stream as fast as it is being
transferring onto/off of the tape.  There are two cases:  an over-run
during write and an over-run during read.  An over-run during writing is
actually not destructive.  The buffer simply loops and duplicate (or
triplicate, etc.) copies of all of the sectors in the buffer are written to
the tape.  As long as this condition can be detected on playback and the
duplicate copies skipped over then there is no harm done (although tens of
kilobytes of tape have been wasted).  The driver detects and corrects this
during playback by watching the sequence number of the sectors being
decoded and skipping over any that are ``earlier'' than the expected
sequence number.

A read over-run is destructive.  In this case data is coming off the tape
faster than the computer can process it and the tape gets a full buffer
length ahead of the computer.  This results in at least one entire buffer
of data (typically tens of kilobytes) being lost.  This kind of error is
detected by seeing a sequence number which is ``later'' than the expected
sequence number.

Unlike data dropouts, both framing and over-run errors are not inherent
problems with the data stream (although framing errors caused by extremely
noisy sector keys will probably be accompanied by badly damaged data
blocks).  A second pass at reading the problem pieces might get the data
off the tape.

In any case this driver design allows for a data recovery solution not
offered by Danmere's own software:  off-line processing.  As discussed
earlier, if the problem is processing speed then the tape data (either raw
or as a \texttt{.tar} file) can be transfered to a disk file and processed
at a leisurely pace.


\section{Hardware}
\label{notes}

\subsection{Preamble}

The following information was obtained by experimenting with a Backer 16
card.  The manual describes it as a model ISA45202 (version 1.2.1).  The
information should also apply to Backer 32 cards since Danmere's web site
suggests that Backer 32s are identical to Backer 16s with the exception of
a different analog section to improve the signal-to-noise ratio.  The names
of the bits on the I/O port are those listed by MONITOR.EXE, an MS-DOS
utility that came with the card.  The rest of the information was obtained
by trial-and-error and might be incorrect/incomplete.


\subsection{Bits on I/O Port}

\begin{center}
\begin{picture}(160,120)
%\put(0,0){\framebox(160,120){}}

\put( 0,108){\makebox(12,12){7}}
\put(12,108){\makebox(12,12){6}}
\put(24,108){\makebox(12,12){5}}
\put(36,108){\makebox(12,12){4}}
\put(48,108){\makebox(12,12){3}}
\put(60,108){\makebox(12,12){2}}
\put(72,108){\makebox(12,12){1}}
\put(84,108){\makebox(12,12){0}}
\multiput(0,96)(12,0){8}{\framebox(12,12){}}

\put(90,90){\line(0,1){ 6}}\put(90,90){\line(1,0){ 6}}
\put(78,78){\line(0,1){18}}\put(78,78){\line(1,0){18}}
\put(66,66){\line(0,1){30}}\put(66,66){\line(1,0){30}}
\put(54,54){\line(0,1){42}}\put(54,54){\line(1,0){42}}
\put(42,42){\line(0,1){54}}\put(42,42){\line(1,0){54}}
\put(30,30){\line(0,1){66}}\put(30,30){\line(1,0){66}}
\put(18,18){\line(0,1){78}}\put(18,18){\line(1,0){78}}
\put( 6, 6){\line(0,1){90}}\put( 6, 6){\line(1,0){90}}

\put(98,90){\makebox(0,0)[l]{Data Rate}}
\put(98,78){\makebox(0,0)[l]{DMA Request}}
\put(98,66){\makebox(0,0)[l]{Data}}
\put(98,54){\makebox(0,0)[l]{Sync}}
\put(98,42){\makebox(0,0)[l]{Frame Busy}}
\put(98,30){\makebox(0,0)[l]{Transmit}}
\put(98,18){\makebox(0,0)[l]{Receive}}
\put(98, 6){\makebox(0,0)[l]{Video Mode}}
\end{picture}
\end{center}

\subsubsection{Data Rate}

Setting this bit to 1 puts 10 bytes on each line of video while setting it
to 0 puts 4 bytes on each line.


\subsubsection{DMA Request}

When writing to the port, the value of this bit starts and stops DMA
transfers.  When reading from the port, the value of this bits indicates
whether or not data is actually being transfered.

Write a 1 to this bit to enable PERIPHERAL \(\longleftrightarrow\) RAM
transfers. Write a 0 to this bit to stop PERIPHERAL \(\longleftrightarrow\)
RAM transfers.  The card is apparently designed for use with the ``demand
transfer'' mode of the DMA controller in which transfers are paused and
resumed asynchronously by the peripheral card raising and lowering its DREQ
line.  It is desireable for peripherals which are, by design, slower than
the system bus to use this mode since it gives other devices access to the
bus while the peripheral catches up to its internal buffer.

When reading from the port this bit indicates the current status of the
DREQ line with a 1 indicating an active DREQ and 0 indicating inactive
DREQ.

I have found that occasionally there is not enough time to read both the
high and low bytes when this bit goes to 0 (while the card pauses between
transfers).  The work-around (to be sure you don't over-run the current
transfer point in the DMA buffer) is to assume the low byte is invalid and
not access memory within 256 bytes of the reported DMA transfer location
(my device driver is written to always stay 512 bytes away just to be
sure).

It is convenient to configure the DMA controller to autoinitialize the
channel that's being used.  This allows the card to loop on its own and
generate a nice continuous video signal without being baby-sat by software.

Writing 0 to the DMA Request bit not only stops the transfer but causes the
card to signal an EOP to the DMA controller which, if configured to
autoinitialize, will reset its address and count registers to the their
original values.


\subsubsection{Data}

I don't know what this does but I believe it mirrors the data as it is
clocked into and out of the card.  Of course, there's no way you can read
the port fast enough to make use of this.


\subsubsection{Sync}

This bit \(= 0\) during both horizontal and vertical sync intervals in the
input signal.  It is (mostly) safe to read the DMA count and residue
registers when this bit is 0.  By ``safe'' I mean that while this bit \(=
1\) the card is actively transfering data so the residue/address read from
the DMA controller is not guaranteed to correspond in any way at all to the
true location of the current transfer.


\subsubsection{Frame Busy}

This bit seems to only be of use during memory \(\longrightarrow\) tape
transfers.  Also, ``frame busy'' is a misnomer;  it would be better to call
this bit ``field busy''.

This bit \(= 0\) while the card is processing a field of video and goes to
1 when it is pausing during the vertical retrace interval.  NOTE:  even if
frame busy \(= 1\) it is still only safe to access the DMA controller while
the Sync bit \(= 0\).  The card has a four byte internal buffer and it
reads one buffer-length ahead of itself in memory.  That means that during
memory \(\longrightarrow\) tape transfers, DMA activity continues a little
after frame busy goes to 1.

During tape \(\longrightarrow\) memory transfers this bit ``rolls''.  The
phase of the 0-to-1 transitions drifts with respect to the vertical sync
making it meaningless.  That is, unless there's some trick to get it to
synchronize but I've tried many things including playing back a tape
recorded using Danmere's own software in the hope that some part of the
data stream was used by the card for this purpose but with no luck.


\subsubsection{Transmit and Receive}

Setting one of these bits to 1 configures the card for the respective
transfer type.  Setting both to 0 disables transfers.  I don't know what
happens if both are set to 1.


\subsubsection{Video Mode}

The meaning of this bit depends on whether the port is being read or
written.
\begin{center}
\begin{tabular}[t]{rl}
writing: & 1 \(\equiv\) NTSC, 0 \(\equiv\) PAL \\
reading: & 1 \(\equiv\) PAL, 0 \(\equiv\) NTSC
\end{tabular}
\end{center}


\subsection{Video mode information}

\subsubsection{NTSC}

In NTSC mode, for even fields data starts on line 10 and ends on line 262;
in odd fields it also starts in line 10 and ends in line 262 but the card
will transfer data corresponding to line 263 of the odd field.  Although
inserted into the video signal, the extra line will not be recorded
properly as it occurs during the 1/2 period line that is at the end of the
odd field. (this has to do with the interlacing of the video image on the
screen: delaying the even fields 1/2 of a line period with respect to odd
fields shifts the phase of the horizontal scanning with respect to the
vertical scanning of the television resulting in a vertical 1/2
line-thickness shift of the lines on the picture tube thus correctly
interlacing the image). Since this last line of the odd field corresponds
to the 1/2 period horizontal scan there is a sync pulse in the middle of
it which overwrites the bits in the middle of the line.

The issue is complicated by the fact that during playback this extra line
may not be reproduced at all --- recall that the video signal standard
does not specify the actual number of lines in a frame but rather the
ratios of the various scan frequencies.  During record this is not a
problem:  the card is responsible for generating the video signal and
generates a precise number of lines per frame.  It seems to me the only
way around the playback ``sloppiness'' is to design a data format which
allows identification of the start of a video field in software (my
current device driver implementation uses this technique).

There are 30 full frames per second (60 fields per second).  If one wants
to insert human-readable graphics into the byte stream then the following
applies.  The order in which data in the DMA buffer is arrange (visually)
on the screen is as expected with the odd field coming first i.e. the byte
display sequence is
\begin{center}
\begin{tabular}[t]{lcl}
line 0   & (odd  field line 10) & --byte0--byte1--byte2--... \\
line 254 & (even field line 10) & --byte0--byte1--byte2--... \\
line 1   & (odd  field line 11) & --byte0--byte1--byte2--... \\
line 255 & (even field line 11) & --byte0--byte1--byte2--... \\
line 2   & (odd  field line 12) & --byte0--byte1--byte2--... \\
etc. ... &&
\end{tabular}
\end{center}
where lines are arranged in numerical order sequentially in memory and the
bits of each byte are arranged msb to lsb left to right on the screen.  
Bits that are 1 appear white, bits that are 0 appear black.  Most
television sets do not \emph{exactly} interlace the video fields.  I don't
know if this is by design or not but that means that line 10 of the even
field lies slightly closer to line 10 of the odd field than it does to
line 11 of the odd field (rather than appearing exactly in the middle).
The consequence for human-readable graphics is that to make a nice stable
image that doesn't flicker, one should make lines in the even field the
same as the corresponding lines in the odd field.  This instead of the
other two options:  using the even field to increase the vertical
resolution or making, for example, line 10 of the even field the same as
line 11 of the odd field (the worst case).

\subsubsection{PAL}

Since I do not have access to PAL video equipment I have not been able to
do any testing of the behaviour of the card in this mode except to check
its output with an oscilloscope.  I, therefore, have far less information.  
In PAL mode, for odd fields active data starts on line 6 and ends on line
310; for even fields data starts on line 319 and ends on line 622 but
again the card transfers an extra line of data corresponding to line 623
of the video signal --- a 1/2 period line.  That gives a total of 610
lines of data transferred per frame with 609 of them containing useful
data.  There are 25 full frames per second (50 fields per second) meaning
PAL has a higher data rate than NTSC but presumably also requires stronger
ECC to guarantee the same overall error rate since both kinds of VCRs use
the same tapes and one generally can't get something for nothing.


\subsection{Behaviour}

The following are some tid-bits regarding the behaviour of the card.
\begin{itemize}
\item
The Sync bit only indicates the status of the input signal, not the output
signal.  Since the round-trip propogation delay through a VCR is quite low,
however, one seems to be able to safetly use this bit as the sync indicator
for the output signal as well.  So, the correct bit to use for
synchronizing updates of the DMA offset is the DMA Request bit but by using
the Sync bit instead one requires that a VCR be connected to the input and
that it be generating a video signal which is a good way of checking the
health of the data path.

\item
During playback, the card will (sometimes) skip over lines containing runs
of 0's.  I do not yet know for sure if there is a pattern to this
behaviour.  After extensive but non-systematic experimentation my current
thinking on this is that the card must see some minimum number of state
transitions per unit time in order for it to keep its bit clock
phase-locked to the data and that this minimum rate is \(\approx\) several
every three bytes.  I have found that ensuring that at least every third
byte is non-zero seems to produce ``Backer-friendly'' data.

\item
The card requires bits on the control port to be set in a particular order
when starting a transfer.  The sequence is a two-step process.  In the
first step, all the desired bit settings are written to the control port
with the exception of the Receive and Transmit bits.  Then, in the second
write, the same bits are set but this time with the transfer direction
included.  For example, to start a low density NTSC memory to tape transfer
you would first write 10000010 = 0x82 followed by 10100010 = 0xA2 to the
control port.

\item
The correct procedure for stopping a transfer is to simply write 0 to the
control port but this should only be done at an inter-frame boundary or the
next transfer might not start up correctly.

\item
When writing, the DMA channel must already be enabled when the transfer is
started as the card will generate a video signal whether or not it's fed
data by the DMA controller.  Any delay will mean that when data does start
getting transfered, its alignment with respect to the video signal will not
be known (although it should be possible to sync the data up again by
carefully watching the Frame Busy bit and the DMA offset register).
\end{itemize}


%\subsection{Windows Format Notes}
%
%This section contains some of the notes I have compiled regarding the data
%format used by Danmere's Windows software.
%\begin{itemize}
%\item
%Unused lines are filled with 0x33.  The exeption is the human-readable file
%number graphic at the bottom of each field which has single lines of 0's
%separating it from the data above and 0x33's below.
%
%\item
%Each field begins with 24 lines filled with 0xE2.  The function of this
%block is not known and its presence does not seem to have any effect on the
%card's behaviour (it does not, for example, fix the rolling of the Frame
%Busy bit during playback).  It's probable that this serves the same role as
%the sector leader used in my own format but I have found that only 8 lines
%are required to satisfy that role.  At least on my equipment...  (reminder:
%a VCR typically requires the first part of the video field to lock onto the
%track so data cannot be reliably recorded in some number of lines at the
%top of each video field).
%\end{itemize}

\end{document}
